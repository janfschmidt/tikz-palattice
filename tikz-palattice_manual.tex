\documentclass[colorlinks]{scrartcl}
\usepackage[english]{babel}
\usepackage[utf8]{inputenc}
\usepackage[T1]{fontenc}
\usepackage{fixltx2e}
\usepackage{graphicx}
\usepackage{longtable}
\usepackage{float}
\usepackage{wrapfig}
\usepackage{rotating}
\usepackage[normalem]{ulem}
\usepackage{amsmath}
\usepackage{textcomp}
\usepackage{marvosym}
\usepackage{wasysym}
\usepackage{amssymb}
\usepackage{hyperref}
\usepackage{siunitx}
\tolerance=1000
\author{Jan Schmidt <schmidt@physik.uni-bonn.de>}
\date{v2.1 (\today)}
\title{tikz-palattice - draw particle accelerator lattices with Ti\textit{k}Z}
\hypersetup{
  pdfkeywords={},
  pdfsubject={},
  pdfcreator={Emacs 24.3.1 (Org mode 8.2.7b)}}
\begin{document}

\maketitle
\begin{figure}[h]
  \centering
  \includegraphics[width=1.1\textwidth]{elsa.pdf}
  \caption{The Electron Stretcher Facility ELSA at Bonn University, drawn with
    tikz-palattice}
\end{figure}

\clearpage
\tableofcontents
\clearpage


\section{Installation}
\label{sec-1}
\subsection{Copy lattice.sty}
\label{sec-1-1}
You just need to copy the lattice.sty file to a place where your \LaTeX{} installation can recognize it.
This can be
\begin{itemize}
\item the same folder as your .tex document
\item in the \LaTeX{} system or user tree
\end{itemize}
e.g. to add it to the system tree for texlive under ubuntu:
\begin{verbatim}
sudo mkdir -p /usr/local/share/texmf/tex/latex/lattice/
sudo cp lattice.sty /usr/local/share/texmf/tex/latex/lattice/
sudo mktexlsr (or sudo texhash)
\end{verbatim}
For this path there is also a Makefile prepared, so just enter
\begin{verbatim}
sudo make install
\end{verbatim}
Otherwise read the documentation of your \LaTeX{} distribution.
\subsection{Required packages}
\label{sec-1-2}
\begin{itemize}
\item tikz, pgf
\item siunitx
\item ifthen
\item xargs
\end{itemize}
\section{What is missing?}
\label{sec-2}
\begin{itemize}
\item manually adding (and maybe changing?) legend entries
\item The look of the elements can be improved
\item More element types can be added
\item \ldots{}
\end{itemize}
\section{Known issues}
\label{sec-3}
\begin{itemize}
\item no error handling implemented
\item no dedicated scoping of internal macros (use of lattice with documentclass standalone recommended)
\item legend entry alignment does not work correctly for some legend scales
\item please report bugs\ldots{}
\end{itemize}

\section{General Remarks}
\label{sec-6}
\begin{itemize}
\item lengths are set in meter, so you write \{1.32\} for \SI{1.32}{\m}.
\item beamline with angle \SI{0}{\degree} goes to the right, positive angles bend counter clockwise.
\item settings (colors, font, rotatelabel,\ldots{}) changed within a \verb+scope+ environment are set back to the previous values outside of \verb+scope+
\item picture scale: for lattice scale=1 an element of 1m length is plotted with 2cm length
\end{itemize}

\section{lattice environment}
\label{sec-4}
To draw a lattice just add
\begin{verbatim}
\usepackage{lattice}
\end{verbatim}
to your preambel and use the lattice environment.
the lattice environment has 2 optional arguments:
\begin{enumerate}
\item \verb+[tikz options]+ give any options for the tikzpicture (e.g. overlay)
\item \verb+[scale]+ scale whole picture (default: 1)
\end{enumerate}
\section{Within lattice environment}
\label{sec-5}
The values in round brackets are default values of optional arguments.
\subsection{Elements}
\label{sec-5-1}
Element names are self-explanatory:

\begin{itemize}
\item \verb+\drift{length/m}[name (none)]+
\item
\begin{verbatim}
\dipole{name}{arc length/m}{bending angle/deg}[type (s)][thickness/m (0.6)]
\end{verbatim}

the \verb+type+ option allows to select different dipole shapes. It can be:
\begin{itemize}
\item br for a bend rectangle magnet (parallel entrance/exit surfaces)
\item r for a rectangle magnet
\item s for a sector magnet (entrance/exit surface 90 degree to beampipe)
\end{itemize}
If you use any other letters, also the default (s) is used.
\item \verb+\quadrupole{name}{length/m}[thickness/m (0.5)]+
\item \verb+\sextupole{name}{length/m}[thickness/m (0.3)]+
\item \verb+\corrector{name}{length/m}[thickness/m (0.25)]+
\item \verb+\kicker{name}{length/m}[thickness/m (0.25)]+
\item \verb+\cavity{name}{length/m}[thickness/m (0.45)]+
\item \verb+\solenoid{name}{length/m}[thickness/m (0.2)]+
\item \verb+\beamdump{name}{length/m}[thickness/m (0.5)]+
\item \verb+\source{name}{length/m}[thickness/m (0.5)]+
\item \verb+\screen{name}[length/m (0.2)]+
\item \verb+\valve{name}+
\item \verb+\marker{name}[length/m (0.35)]+ a line perpendicular to beamline of given length
\end{itemize}
\subsection{Modify your lattice/elements}
\label{sec-5-2}
\begin{itemize}
\item \verb+\rotate{angle/deg}+ "bends" the beamline., offset from current angle
\item \verb+\setangle{angle/deg}+ "bends" the beamline., set absolute angle
\item \verb+\goto{coordinate name}+ sets current position and angle to values saved with \verb+\savecoordinate+
hint: e.g. use to draw injection/extraction
\item \verb+\start{coordinate/m}+ sets starting point of lattice. use before first element
coordinate in form (x,y) or any tikz label, e.g. (mylabel.east)
hint: use with \verb+\savecoordinate+ to connect lattices! (compile twice!)
\item \verb+\drawrule{position/m}[tick distance/m (1)]+\ldots{}\\ \verb+[scale (1)][height/m (0.1)]+\\  a rule to visualize lattice size. coordinate in form (x,y) or any tikz label, e.g. (mylabel.east)
\item \verb+\legend{position/m}[scale (1)]+ a legend with all element types that occur in the lattice before this command.
position is north west (upper left corner) of the legend box.
the scale option scales the whole box including the text, which has the usual label textsize for scale=1
\item \verb+\completelegend{position/m}[scale (1)]+ similar to \verb+\legend+, but shows all existing element types.
\end{itemize}
\subsubsection{Labels}
\label{sec-5-2-1}
\begin{itemize}
\item \verb+\turnlabels+ moves labels to other side of elements (swap with marker labels)
\item \verb+\rotatelabels{angle/deg}[anchor (automatic)]+ allows rotation of element labels.
the anchor sets the center of rotation (north, center, south west, \ldots{}). west corresponds to labels first character.
\item environment \texttt\{labeldistance\{distance/m\}\} sets distance of text labels to element center for all elements within this environment (default is 0.35)
\item \verb+\setlabeldistance{fontsize}+ sets distance of text labels to element center for following elements (default is 0.35)
\item \verb+\resetlabeldistance{fontsize}+ resets distance of text labels to element center for following elements to default 0.35
\item \verb+\setlabelfont{fontsize}+ text label fontsize (default is \verb+\normalsize+)
\end{itemize}
\subsubsection{Colors}
\label{sec-5-2-2}
The color can be changed at any point in the lattice. A setting is valid until the next color setting comand.
\begin{itemize}
\item \verb+\setlabelcolor{color}+ for textlabels (set to white to hide labels).
\item \verb+\setlinecolor{type}{color}+ for type drift and marker.
\item \verb+\setelementcolor{type}{color}[gradient color (white)]+ for all element types. set gradient color = color to "disable" gradient
\item \verb+\resetlinecolor{type}+ reset to default color.
\item \verb+\resetelementcolor{type}+ reset to default color.
\item environment \verb+\begin{fade}[opacity (0.25)]+ sets the opacity of all elements within the environment to fade out regions of the lattice - e.g. for presentations. Also sets all colors to gray.
This can also be used to completely hide regions by setting opacity to zero.
\end{itemize}
\subsection{Access lattice coordinates}
\label{sec-5-3}
You can use element coordinates to draw anything you want using pgf/tikz. You can even connect lattices to draw injection/extraction or a complete accelerator facility.
\begin{itemize}
\item \verb+\savecoordinate{name}[position (east)]+ saves coordinate of previous element
to access it later.
\begin{itemize}
\item position specifies the exact place of the element. East (default) and center are available. East is always downstream.
\item you can use all tikz/pgf commands within lattice environment to draw anything.
\item You can use this to connect multiple beamlines within a lattice environment with \verb+\goto{name}+
\item You can use this to connect multiple lattices with \verb+\start{name}+. use tikz overlay option (1. argument of lattice)
\item see example 3
\end{itemize}
\end{itemize}
\end{document}