%% example3_coordinates.tex
%% Copyright 2015 J. Schmidt
%% tikz-palattice LaTeX package
% 
% This work may be distributed and/or modified under the
% conditions of the LaTeX Project Public License, either version 1.3
% of this license or (at your option) any later version.
% The latest version of this license is in
% http://www.latex-project.org/lppl.txt
% and version 1.3 or later is part of all distributions of LaTeX
% version 2005/12/01 or later.
% 
% This work has the LPPL maintenance status `maintained'.
% 
% The Current Maintainer of this work is J. Schmidt.
% 
% This work consists of the files tikz-palattice.sty and tikz-palattice_documentation.tex
% and the 5 example files example1_linear.tex, example2_circular.tex,
% example3_coordinates.tex, example4_labels.tex and elsa.tex

\documentclass[a4paper]{scrartcl}
\usepackage[english]{babel}
\usepackage[utf8]{inputenc}
\usepackage{subcaption}
\usepackage{verbatim}
\usepackage{tikz-palattice}


% compile twice to connect lattices!


\title{Two more lattices}
\subtitle{scaling lattices and accessing coordinates}
\date{}

\begin{document}
\maketitle

\begin{figure}[h]
\centering
\begin{subfigure}{.45\textwidth}
  \centering
  % this is the ring from example 2, but plotted smaller (30%).
  % the element sizes (in meter) are the same - as the rule indicates.
  \begin{lattice}[0.3]
    \setlabelfont{\tiny}
    \begin{labeldistance}{0.3}
      \drift{1}
      \dipole{M1}{2}{45}
      \drift{1}
      \turnlabels
      \dipole{M2}{2}{45}
      \turnlabels
      \drift{1}
      \dipole{M3}{2}{45}
      \drift{1}
      \savecoordinate{myCoord1}[center]
      \dipole{M4}{2}{45}
      \drift{1}
      \dipole{M5}{2}{45}
      \drift{1}
      \dipole{M6}{2}{45}
      \drift{1}
      \savecoordinate{myCoord2}[center]
      \dipole{M7}{2}{45}
      \drift{1}
      \dipole{M8}{2}{45}
      \drawrule{(-2.5,-1)}[2]
    \end{labeldistance}
    % custom drawing using saved coordinates (to center of drifts)
    \draw[->, thick] (myCoord2) -- (myCoord1) node[midway,fill=white,rotate=20] {signal};
  \end{lattice}
\end{subfigure}
%
%
%
\begin{subfigure}{.45\textwidth}
  \centering
  \begin{lattice}
    \setangle{70}
    \northlabels %labels to the left
    \drift{0.5}
    \quadrupole{QF1}{0.3}
    \drift{0.2}
    \kicker{}{0.1}
    \drift{0.07}
    \quadrupole{QD2}{0.3}
    \drift{0.4}
    \setelementcolor{kicker}{cyan}
    \kicker{Septum}{0.5}
    \resetelementcolor{kicker}
    \savecoordinate{Septum}[center] %save to connect other beamline
    \drift{0.8}
    \kicker{}{0.1}
    \drift{0.37}
    \quadrupole{QD3}{0.2}
    \drift{0.17}
    \kicker{}{0.1}
    \drift{0.1}
    \sextupole{SF1}{0.2}
    \drift{0.27}
    \dipole{MB1}{0.2}{30}[s][0.4]
    \drift{0.2}

    % now go back to saved coordinate and continue with another beamline
    \goto{Septum}
    \rotate{-20}
    \southlabels %labels to the right
    \drift{0.6}
    \kicker{SS1}{0.2}
    \drift{0.2}
    \kicker{SSH1}{0.2}
    \drift{0.3}
    \dipole{M1}{0.2}{-30}[r][0.4]
    \begin{fade} %fade out this part
      \drift{0.2}
      \dipole{M2}{0.2}{50}[r][0.4]
      \drift{0.2}
      \dipole{M3}{0.2}{-20}[r][0.4]
      \drift{0.7}
    \end{fade}

    \drawrule{(-0.75,-0.2)}[0.8]
  \end{lattice}
\end{subfigure}
\caption{A figure with two lattices}
\label{fig:lattices}
\end{figure}






\clearpage
\begin{figure}
\centering
\begin{subfigure}[b]{.45\textwidth}
\centering
 \begin{lattice}
    \rotate{90}
    \northlabels %labels to the left
    \drift{0.5}
    \quadrupole{QF1}{0.3}
    \drift{0.2}
    \kicker{}{0.1}
    \drift{0.07}
    \quadrupole{QD2}{0.3}
    \drift{0.4}
    \setelementcolor{kicker}{cyan}
    \kicker{Septum}{0.5}
    \resetelementcolor{kicker}
    \savecoordinate{Septum}[center] %save to connect other beamline
    \drift{0.8}
    \kicker{}{0.1}
    \drift{0.37}
    \quadrupole{QD3}{0.2}
    \drift{0.17}
    \kicker{}{0.1}
    \drift{0.1}
    \sextupole{SF1}{0.2}
    \drift{0.27}
    \dipole{MB1}{0.2}{30}[s][0.4]
    \drift{0.2}
  \end{lattice}
  % 
  %now draw second lattice and connect them using the saved coordinate
  \begin{lattice}[1][overlay]
    \start{(Septum)}
    \rotate{70}
    \drift{0.6}
    \kicker{SS1}{0.2}
    \drift{0.2}
    \kicker{SSH1}{0.2}
    \drift{0.3}
    \dipole{M1}{0.2}{-30}[r][0.4]
    \begin{fade} %fade out this part
      \drift{0.2}
      \dipole{M2}{0.2}{50}[r][0.4]
      \drift{0.2}
      \dipole{M3}{0.2}{-20}[r][0.4]
      \drift{0.7}
    \end{fade}
    \legend{(-1,6.5)}
  \end{lattice}
 \caption{two lattice environments and start}
\end{subfigure}
%
\begin{subfigure}[b]{.45\textwidth}
  \centering
  \begin{lattice}
    \rotate{90}
    \northlabels %labels to the left
    \drift{0.5}
    \quadrupole{QF1}{0.3}
    \drift{0.2}
    \kicker{}{0.1}
    \drift{0.07}
    \quadrupole{QD2}{0.3}
    \drift{0.4}
    \setelementcolor{kicker}{cyan}
    \kicker{Septum}{0.5}
    \resetelementcolor{kicker}
    \savecoordinate{Septum}[center] %save to connect other beamline
    \drift{0.8}
    \kicker{}{0.1}
    \drift{0.37}
    \quadrupole{QD3}{0.2}
    \drift{0.17}
    \kicker{}{0.1}
    \drift{0.1}
    \sextupole{SF1}{0.2}
    \drift{0.27}
    \dipole{MB1}{0.2}{30}[s][0.4]
    \drift{0.2}

    % now go back to saved coordinate and continue with another beamline
    \goto{Septum}
    \rotate{-20}
    \southlabels %labels to the right
    \drift{0.6}
    \kicker{SS1}{0.2}
    \drift{0.2}
    \kicker{SSH1}{0.2}
    \drift{0.3}
    \dipole{M1}{0.2}{-30}[r][0.4]
    \begin{fade} %fade out this part
      \drift{0.2}
      \dipole{M2}{0.2}{50}[r][0.4]
      \drift{0.2}
      \dipole{M3}{0.2}{-20}[r][0.4]
      \drift{0.7}
    \end{fade}
    \legend{(0,6.5)}
  \end{lattice}
  \caption{one lattice environment and goto}
\end{subfigure}
\caption{Two ways to connect lattices}
\label{fig:twoways}
\end{figure}

\end{document}
