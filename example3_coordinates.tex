\documentclass[a4paper]{scrartcl}
\usepackage[english]{babel}
\usepackage[utf8]{inputenc}
\usepackage{subcaption}
\usepackage{lattice}


% compile twice to connect lattices!


\title{Two more lattices}
\subtitle{scaling lattices and accessing coordinates}
\date{}

\begin{document}
\maketitle

\begin{figure}[h]
\centering
\begin{subfigure}{.45\textwidth}
  \centering
  % this is the ring from example 2, but plotted smaller (30%).
  % the element sizes (in meter) are the same - as the rule indicates.
  \begin{lattice}[][0.3]
    \setlabelfont{\tiny}
    \begin{labeldistance}{0.2}
      \drift{1}
      \dipole{M1}{2}{45}
      \drift{1}
      \turnlabels
      \dipole{M2}{2}{45}
      \turnlabels
      \drift{1}
      \dipole{M3}{2}{45}
      \drift{1}
      \savecoordinate{myCoord1}[center]
      \dipole{M4}{2}{45}
      \drift{1}
      \dipole{M5}{2}{45}
      \drift{1}
      \dipole{M6}{2}{45}
      \drift{1}
      \savecoordinate{myCoord2}[center]
      \dipole{M7}{2}{45}
      \drift{1}
      \dipole{M8}{2}{45}
      \drawrule{(-2.5,-1)}[2]
    \end{labeldistance}
    % custom drawing using saved coordinates (to center of drifts)
    \draw[->, thick] (myCoord2) -- (myCoord1) node[midway,fill=white,rotate=20] {signal};
  \end{lattice}
\end{subfigure}
%
%
%
\begin{subfigure}{.45\textwidth}
  \centering
  \begin{lattice}
    \rotate{90}  %start upwards
    \turnlabels %labels to the left
    \drift{0.5}
    \quadrupole{QF1}{0.3}
    \drift{0.2}
    \kicker{}{0.1}
    \drift{0.07}
    \quadrupole{QD2}{0.3}
    \drift{0.4}
    \setelementcolor{dipole}{cyan}
    \dipole{Septum}{0.5}{0}[r]
    \savecoordinate{Septum} %save to connect other beamline
    \drift{0.8}
    \kicker{}{0.1}
    \drift{0.37}
    \quadrupole{QD3}{0.2}
    \drift{0.17}
    \kicker{}{0.1}
    \drift{0.1}
    \sextupole{SF1}{0.2}
    \drift{0.27}

    \drawrule{(-0.75,-0.2)}[0.5]
  \end{lattice}

  % now draw second lattice and connect them using the saved coordinate
  \begin{lattice}[overlay]
    \start{(Septum)}
    \rotate{70}
    \drift{0.6}
    \kicker{SS1}{0.2}
    \drift{0.2}
    \kicker{SSH1}{0.2}
    \drift{0.3}
    \dipole{M1}{0.2}{-30}[r]
    \begin{fade} %fade out this part
      \drift{0.2}
      \dipole{M2}{0.2}{50}[r]
      \drift{0.2}
      \dipole{M3}{0.2}{-20}[r]
      \drift{0.7}
    \end{fade}
  \end{lattice}
\end{subfigure}
\caption{A figure with two lattices}
\label{fig:lattices}
\end{figure}

\end{document}
